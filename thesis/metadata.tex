%%% Please fill in basic information on your thesis, which will be automatically
%%% inserted at the right places. You need to replace \xxx{...} by real data.

% Type of your thesis:
%	"bc" for Bachelor's
%	"mgr" for Master's
%	"phd" for PhD
%	"rig" for rigorosum
\def\ThesisType{bc}

% Language of your study programme:
%	"cs" for Czech
%	"en" for English
\def\StudyLanguage{cs}

% Thesis title in English (exactly as in the official assignment)
% (Note: \xxx is a "ToDo label" which makes the unfilled visible. Remove it.)
\def\ThesisTitle{Traffic Signal Optimization}

% Author of the thesis (you)
\def\ThesisAuthor{David Ruda}

% Year when the thesis is submitted
\def\YearSubmitted{2025}

% Name of the department or institute, where the work was officially assigned
% (according to the Organizational Structure of MFF UK in English,
% see https://www.mff.cuni.cz/en/faculty/organizational-structure,
% or a full name of a department outside MFF)
\def\Department{Department of Theoretical Computer Science and Mathematical Logic}

% Is it a department (katedra), or an institute (ústav)?
\def\DeptType{Department}

% Thesis supervisor: name, surname and titles
\def\Supervisor{RNDr. Jiří Fink, Ph.D.}

% Supervisor's department (again according to Organizational structure of MFF)
\def\SupervisorsDepartment{Department of Theoretical Computer Science and Mathematical Logic}

% Study programme (does not apply to rigorosum theses)
\def\StudyProgramme{Computer Science}

% An optional dedication: you can thank whomever you wish (your supervisor,
% consultant, who provided you with tea and pizza, etc.)
\def\Dedication{%
I would like to thank my supervisor, \Supervisor, for his patience, guidance, and valuable feedback throughout the development of this thesis. \\
I would also like to thank my family and friends for their unwavering support and encouragement during this journey.

% MetaCentrum acknowledgement
Computational resources were provided by the e-INFRA CZ project (ID:90254), supported by the Ministry of Education, Youth and Sports of the Czech Republic.

% LLM acknowledgement
The clarity and readability of this thesis were enhanced using tools based on Large Language Models (LLMs).
}

% Abstract (recommended length around 80-200 words; this is not a copy of your thesis assignment!)
\def\Abstract{%
As cities grow and traffic congestion worsens, traffic signal optimization is becoming an increasingly important real-world problem for ensuring efficient urban mobility.
This thesis explores a simplified version of this problem, originally presented in the Google Hash Code competition.
The task involves optimizing the schedules of traffic lights at city intersections to maximize the number of cars reaching their destinations before a deadline, while minimizing the overall time spent in traffic.
We develop a fast and efficient simulator to evaluate solutions for the task.
We then integrate this simulator into an optimization pipeline with three heuristic algorithms: Genetic Algorithm, Hill Climbing, and Simulated Annealing. We experimentally compare these algorithms on the provided datasets, achieving new best scores on two datasets.
}

% 3 to 5 keywords (recommended) separated by \sep
% Keywords are useful for indexing and searching for the theses by topic.
\def\ThesisKeywords{%
traffic signal optimization \sep Google Hash Code \sep genetic algorithm \sep hill climbing \sep simulated annealing
}

% If any of your metadata strings contains TeX macros, you need to provide
% a plain-text version for use in XMP metadata embedded in the output PDF file.
% If you are not sure, check the generated thesis.xmpdata file.
\def\ThesisAuthorXMP{\ThesisAuthor}
\def\ThesisTitleXMP{\ThesisTitle}
\def\ThesisKeywordsXMP{\ThesisKeywords}
\def\AbstractXMP{\Abstract}

% If your abstracts are long and do not fit in the infopage, you can make the
% fonts a bit smaller by this setting. (Also, you should try to compress your abstract more.)
\def\InfoPageFont{}
%\def\InfoPageFont{\small}  % uncomment to decrease font size

% If you are studing in a Czech programme, you also need to provide metadata in Czech:
% (in English programmes, this is not used anywhere)

\def\ThesisTitleCS{Optimalizace světelných křižovatek}
\def\DepartmentCS{Katedra teoretické informatiky a matematické logiky}
\def\DeptTypeCS{Katedra}
\def\SupervisorsDepartmentCS{Katedra teoretické informatiky a matematické logiky}
\def\StudyProgrammeCS{Informatika}

\def\ThesisKeywordsCS{%
optimalizace světelných křižovatek \sep Google Hash Code \sep genetický algoritmus \sep horolezecký algoritmus \sep simulované žíhání
}

\def\AbstractCS{%
S rostoucí velikostí měst a zhoršující se dopravní situací je optimalizace světelných křižovatek stále důležitějším problémem pro zajištění efektivní dopravy ve městech. Tato práce zkoumá zjednodušenou verzi tohoto reálného problému, která byla původně zadána v soutěži Google Hash Code.
Cílem je optimalizovat nastavení semaforů na městských křižovatkách tak, aby co nejvíce aut dorazilo do cíle před stanoveným časovým limitem a zároveň se minimalizoval celkový čas strávený v dopravě.
Vyvíjíme rychlý a efektivní simulátor, který umožňuje vyhodnocovat navržená řešení pro tento problém. S pomocí tohoto simulátoru následně optimalizujeme nastavení semaforů pomocí tří heuristických algoritmů: genetického algoritmu, horolezeckého algoritmu a simulovaného žíhání.
Tyto algoritmy experimentálně porovnáváme na zadaných datových sadách a dosahujeme nových nejlepších výsledků na dvou z nich.
}
