%%% The main file. It contains definitions of basic parameters and includes all other parts.

% Meta-data of your thesis (please edit)
\input metadata.tex

% Generate metadata in XMP format for use by the pdfx package
\input xmp.tex

%% Settings for single-side (simplex) printing
% Margins: left 40mm, right 25mm, top and bottom 25mm
% (but beware, LaTeX adds 1in implicitly)
\documentclass[12pt,a4paper]{report}
\setlength\textwidth{145mm}
\setlength\textheight{247mm}
\setlength\oddsidemargin{15mm}
\setlength\evensidemargin{15mm}
\setlength\topmargin{0mm}
\setlength\headsep{0mm}
\setlength\headheight{0mm}
% \openright makes the following text appear on a right-hand page
\let\openright=\clearpage

%% Settings for two-sided (duplex) printing
% \documentclass[12pt,a4paper,twoside,openright]{report}
% \setlength\textwidth{145mm}
% \setlength\textheight{247mm}
% \setlength\oddsidemargin{14.2mm}
% \setlength\evensidemargin{0mm}
% \setlength\topmargin{0mm}
% \setlength\headsep{0mm}
% \setlength\headheight{0mm}
% \let\openright=\cleardoublepage

%% If the thesis has no printed version, symmetric margins look better
% \documentclass[12pt,a4paper]{report}
% \setlength\textwidth{145mm}
% \setlength\textheight{247mm}
% \setlength\oddsidemargin{10mm}
% \setlength\evensidemargin{10mm}
% \setlength\topmargin{0mm}
% \setlength\headsep{0mm}
% \setlength\headheight{0mm}
% \let\openright=\clearpage

%% Generate PDF/A-2u
\usepackage[a-2u]{pdfx}

%% Prefer Latin Modern fonts
\usepackage{lmodern}
% If we are not using LuaTeX, we need to set up character encoding:
\usepackage{iftex}
\ifpdftex
\usepackage[utf8]{inputenc}
\usepackage[T1]{fontenc}
\usepackage{textcomp}
\fi

%% Further useful packages (included in most LaTeX distributions)
\usepackage{amsmath}        % extensions for typesetting of math
\usepackage{amsfonts}       % math fonts
\usepackage{amsthm}         % theorems, definitions, etc.
\usepackage{bm}             % boldface symbols (\bm)
\usepackage{booktabs}       % improved horizontal lines in tables
\usepackage{caption}        % custom captions of floating objects
\usepackage{dcolumn}        % improved alignment of table columns
\usepackage{floatrow}       % custom float environments
\usepackage{graphicx}       % embedding of pictures
\usepackage{indentfirst}    % indent the first paragraph of a chapter
\usepackage[nopatch=item]{microtype}   % micro-typographic refinement
\usepackage{paralist}       % improved enumerate and itemize
\usepackage[nottoc]{tocbibind} % makes sure that bibliography and the lists
			    % of figures/tables are included in the table
			    % of contents
\usepackage{xcolor}         % typesetting in color

% Define the same colors as in my plots
% https://www.overleaf.com/learn/latex/Using_colors_in_LaTeX#Creating_your_own_colors
\definecolor{myblue}{RGB}{68, 117, 177}
\definecolor{myorange}{RGB}{207, 134, 59}
\definecolor{mygreen}{RGB}{89, 144, 65}
\definecolor{myred}{RGB}{172, 72, 66}

% The hyperref package for clickable links in PDF and also for storing
% metadata to PDF (including the table of contents).
% Most settings are pre-set by the pdfx package.
\hypersetup{unicode}
\hypersetup{breaklinks=true}

% Packages for computer science theses
% https://tex.stackexchange.com/questions/145760/how-may-i-remove-endif-and-endfor-in-algorithmicx/145764#145764
%\usepackage{algpseudocode}  % part of algorithmicx package
\usepackage[noend]{algpseudocode}  % use noend to not display endif, endfor, etc.

% https://latex.org/forum/viewtopic.php?t=15388
%\usepackage{algorithm}
\usepackage[plain]{algorithm} % use plain to remove horizontal lines in algorithm

\usepackage{fancyvrb}       % improved verbatim environment
\usepackage{listings}       % pretty-printer of source code

% You might want to use cleveref for references
% \usepackage{cleveref}

% Set up formatting of bibliography (references to literature)
% Details can be adjusted in macros.tex.
%
% BEWARE: Different fields of research and different university departments
% have their own customs regarding bibliography. Consult the bibliography
% format with your supervisor.
%
% The basic format according to the ISO 690 standard with numbered references
\usepackage[natbib,style=iso-numeric,sorting=none]{biblatex}
% ISO 690 with alphanumeric references (abbreviations of authors' names)
%\usepackage[natbib,style=iso-alphabetic]{biblatex}
% ISO 690 with references Author (year)
%\usepackage[natbib,style=iso-authoryear]{biblatex}
%
% Some fields of research prefer a simple format with numbered references
% (sorting=none tells that bibliography should be listed in citation order)
%\usepackage[natbib,style=numeric,sorting=none]{biblatex}
% Numbered references, but [1,2,3,4,5] is compressed to [1-5]
%\usepackage[natbib,style=numeric-comp,sorting=none]{biblatex}
% A simple format with alphanumeric references:
%\usepackage[natbib,style=alphabetic]{biblatex}

% Use this to fix overfull hboxes in the bibliography
% https://tex.stackexchange.com/questions/171999/overfull-hbox-in-biblatex
\emergencystretch=1em

% Load the file with bibliography entries
\addbibresource{bibliography.bib}

% Our definitions of macros (see description inside)
\input macros.tex

% Include automatic generation of abbreviations
% \usepackage{glossaries-extra}
% \makeglossaries
% \input abbreviations.tex

% Change these to use sentence case
\renewcommand{\listfigurename}{List of figures}
\renewcommand{\listtablename}{List of tables}

%%% Title page and various mandatory informational pages
\begin{document}
\include{title}

%%% A page with automatically generated table of contents of the thesis

\tableofcontents

%%% Each chapter is kept in a separate file
\chapter*{Introduction}
\addcontentsline{toc}{chapter}{Introduction}

%Zadání
%
%Simulace problému z Google Hash Code 2021 a jeho optimalizace pomocí různých technik umělé inteligence Genetický algoritmus, Local search, Simulated annealing a porovnání těchto přístupů
%
%Cílem je efektivně implementovat simulaci a vyzkoušet různé algoritmy pro její optimalizaci, nikoliv implementovat samotné optimalizační algoritmy (avšak pokud bude vhodnější genetické algoritmy/local search ručně naimplementovat než použít knihovnu, bude to možné, ale není to cílem)
%
%
% Zásady pro vypracování
% Student se ve své práci bude věnovat řízení množiny křižovatek zadané grafem dle specifikace Google Hash Code [1].
% Student práci začne implementací vlastního simulátoru křižovatek dle uvedené specifikace, který bude používat k optimalizaci řízení.
% K optimalizaci student s použitím vhodných nástrojů implementuje a následně experimentálně porovná následující algoritmy.
% 1) Evoluční algoritmy
% 2) Lokální prohledávání
% Na základě získaných zkušeností student dále zvolí některý z následujících přístupů.
% 3) Memetiské algoritmy
% 4) Metaheuristiky
% 5) Náhradní modely

Due to its significant impact on urban mobility and the environment, Traffic Signal Control (TSC) is a widely studied problem~\cite{zhao2012computational}. As the number of vehicles on the road continues to increase~\cite{caves2004encyclopedia}, its importance is growing even further.
Traffic signal optimization is known to be a cost-effective method for reducing congestion without physically changing the road infrastructure~\cite{wang2021traffic}.
Optimized traffic lights reduce delays at intersections~\cite{mu2022traffic}, which directly translates into time savings for commuters and improves the overall efficiency of transportation networks.
Additionally, TSC plays a critical role in reducing vehicle emissions~\cite{gunarathne2023traffic}, helping to reduce pollution and promote environmental sustainability.

There are many approaches to solving TSC~\cite{qadri2020state}. Early static, fixed-time designs have evolved into real-time adaptive systems and data-driven algorithms, utilizing data from various detectors and sensors. In practice, commonly used are adaptive traffic control systems (ATCS) such as SCOOT, SCATS, and SURTRAC~\cite{smith2013surtrac}, which continuously adjust splits, offsets, and cycle lengths network-wide. In addition, optimization-based methods---such as Genetic Algorithm~\cite{costa2020intersection}, Simulated Annealing~\cite{qadri2020state}, and Model Predictive Control~\cite{ye2019survey}---have also been employed for traffic signal planning and coordination. More recently, learning-based methods, particularly those based on Reinforcement Learning and Deep Reinforcement Learning, are increasingly being explored as alternatives to traditionally used methods~\cite{zhao2024survey, saadi2025survey}.

In this thesis, we focus on the Traffic signaling problem from the Google Hash Code competition~\cite{google2023google}. It serves as a simplified version of the real-world problem of traffic signal optimization in a city. First, we implement a fast and efficient C++ simulator for the problem and wrap it as a Python package to enable easy use and integration with the broader Python ecosystem, without compromising on performance.
We then utilize the simulator as a black-box fitness function for three heuristic algorithms to optimize the traffic light schedules.

Hill Climbing (HC) is the simplest of the methods and it has been used by some participants both during and after the competition.
Genetic Algorithm (GA) is a more complex method that, to the best of our knowledge, has not been applied to this particular competition problem before.
As a third method, we choose Simulated Annealing (SA), a metaheuristic that can be viewed as a simple extension of HC. However, it appeared to be a suitable choice after initial tests suggested that GA's broader search capabilities may not be so beneficial for this problem.
We then experimentally compare the performance of these algorithms on the provided competition datasets, which vary in size and structure.

The thesis is structured as follows:
Chapter~\ref{chap:problem_description} presents the Traffic signaling problem in detail, along with preprocessing steps, datasets, and the simulator.
Chapter~\ref{chap:optimization_methods} covers the theory of the optimization methods used in the thesis.
Chapter~\ref{chap:algorithms_application} describes the application of the optimization methods to our specific problem, including initialization, algorithm operators, and hyperparameters.
Chapter~\ref{chap:experimental_results} presents the experimental results comparing the performance of the algorithms on the provided datasets.
Appendix~\ref{chap:user_guide} provides a user guide detailing how to use the simulator, run optimization, and execute the scripts replicating our experiments.
Appendix~\ref{chap:developer_documentation} contains developer documentation briefly describing implementation details of the simulator and optimization.

\chapter{Problem description} \label{chap:problem_description}

This chapter introduces the optimization problem addressed in this thesis. Section~\ref{sec:competition_overview} provides a brief overview of the competition setting where the problem was initially assigned. In Section~\ref{sec:original_problem_statement}, the original problem statement is presented and explained in detail.
Section~\ref{sec:preprocessing} explains preprocessing steps that we apply to simplify the problem for optimization.
Section~\ref{sec:datasets} describes the datasets provided with the problem. Finally, Section~\ref{sec:simulator} presents the custom simulator tool that we developed to evaluate solutions to the problem.

\section{Competition overview} \label{sec:competition_overview}

The problem we address in this thesis was originally assigned in the qualifying round of Google Hash Code 2021. Google Hash Code was a global team programming competition, running between 2014--2022, where teams of 2--4 competitors solved an optimization problem with the goal of achieving the best score in a limited time of 4 hours. Despite its discontinuation, the competition remains a valuable resource for a wide range of optimization problems and has been the subject of various follow-up studies and articles~\cite{rodrigues2023principled, li2022building}.

The usual solution procedure for our problem was as follows: Read the input data, create a trivial solution and write it to the output file in the specified format, and upload the solution to the evaluation system. The evaluation system not only displayed the score, but also some informative statistics, such as the number of cars that reached the finish before the deadline, the cars that arrived earliest and latest, the average cycle length of traffic lights at intersections, etc.
For some datasets, an interactive visualization of the evaluation process was also available, allowing to see the structure of a particular dataset. Although a local simulator was not needed to solve the problem, the manual upload of the solution to the evaluation system was slow and cumbersome and therefore not suitable for the use of efficient optimization algorithms.

Most teams were able to construct trivial solutions, which they then tried to improve by randomly changing the values. However, the best teams were able to write their own local simulator and use it to run multiple heuristics to get a better total score. Still, there was no time for anything more complex than a simple random search.

The total score was the sum of the scores of all 6 datasets (A--F). The first dataset (A) served as a ``toy problem'' mainly for debugging purposes, but the rest of the datasets were large enough for optimization. It should be noted that the distribution of points among the datasets is uneven, so the contestants mostly focused on the datasets with the highest possible score (D, F) and pragmatically skipped optimizing the rest (B, C, E).

\section{Original problem statement} \label{sec:original_problem_statement}

The full problem statement\footnote{\url{https://github.com/google/coding-competitions-archive/blob/main/hashcode/hashcode_2021_qualification_round.pdf}} is available in the Google Coding Competitions archive~\cite{google2023google}. Here we describe only the parts that are important for understanding this work.
In short, the task is as follows:
\begin{quote}
    \textit{Given a city plan describing intersections and streets and cars with planned paths through the city, optimize the schedule of traffic lights to minimize the total amount of time spent in traffic, and help as many cars as possible reach their destination before a specified deadline.}
\end{quote}
In terms of graph theory, the city plan is a \textit{directed graph}. The intersections are \textit{vertices} and the streets are \textit{directed edges} (see Figure~\ref{fig:hashcode_city_plan}). The planned path for each car is indeed a \textit{path} in this graph because it has a different start and end, and no intersection is repeated.

\begin{figure}
    \centering
    \includegraphics[width=\linewidth]{img/hashcode/figure1.png}
    %\includegraphics[width=.8\linewidth]{img/hashcode/figure1.png}
    \caption[Example of a city plan]{
        Example of a city plan \cite{google2023google}.
    }
    \label{fig:hashcode_city_plan}
\end{figure}

\subsection{Streets and intersections}

In the city, we have a set of intersections
$I$, where $\abs{I} \in [2, 10^5]$,
and a set of streets
$S \subseteq \{(u, v) | u,v \in I \land u \neq v\}$, where $\abs{S} \in [2, 10^5]$.
Each street $s \in S$ is a unique one-way connection between two different intersections $u$, $v$; two distinct streets $(u, v)$ and $(v, u)$ in opposite directions between the same two intersections are allowed. Each street $s \in S$ has a fixed time $l(s) \in \mathbb{N}_+$ that it takes a car to get from the beginning to the end of the street, independently of the other cars on the street.
Each intersection $i \in I$ has a set of incoming streets $S_i^+ \subset S$, where $\abs{S_i^+} \geq 1$, and a set of outgoing streets $S_i^- \subset S$, where $\abs{S_i^-} \geq 1$; thus each intersection has at least one incoming street and at least one outgoing street.

\subsection{Traffic lights and schedules}

In each intersection $i \in I$, there is a traffic light at the end of each \textit{incoming} street $s^+ \in S_i^+$. The traffic light has two states---green and red. Green means the cars from this street can pass through the intersection and continue to any \textit{outgoing} street $s^- \in S_i^-$ in their path. Red means the cars must stop until the light turns green again. At most one traffic light can be green at each intersection at any time.

When the light is red, cars arriving at the end of a street queue up and wait for the light to turn green. The queue does not take up any space and does not change the distance cars have to travel. When the light is green, one car can pass through an intersection every second. Passing through an intersection, i.e., moving from the end of an incoming street to the beginning of an outgoing street, takes no additional time.

For each intersection $i \in I$, we can set a traffic light schedule. This schedule determines the order and duration of green light for the incoming streets of the intersection. The schedule repeats in a cycle until the end of the simulation (see Figure~\ref{fig:hashcode_traffic_lights}). Each street can appear at most once in the schedule. If a street is not included in the schedule, it is red the whole time, and any waiting cars are blocked. By default, intersections have no schedule and all streets are red.

% https://tex.stackexchange.com/questions/69869/image-taking-up-full-page
\begin{figure}[ht] % h = here, t = top, p = page of floats
    \centering
    \includegraphics[width=\linewidth]{img/hashcode/figure2-abc.png}
    % \includegraphics[width=.8\linewidth]{img/hashcode/figure2-abc.png}
    \includegraphics[width=\linewidth]{img/hashcode/figure2-def.png}
    % \includegraphics[width=.8\linewidth]{img/hashcode/figure2-def.png}
    \caption[Example of a traffic light schedule]{
        This figure shows how the traffic light schedule works for an intersection with two incoming streets \cite{google2023google}.
        The schedule is as follows: First \textit{a-street} for $2$ seconds, then \textit{b-street} for $3$ seconds.
        We can see the first two cars from \textit{a-street} pass in the first two seconds, then the green light switches to \textit{b-street} for three seconds,
        allowing the two cars from \textit{b-street} pass. The last car from \textit{a-street} waits till the beginning of the next cycle and then passes.
    }
    \label{fig:hashcode_traffic_lights}
\end{figure}

\subsection{Cars} \label{subsec:cars}

Furthermore, we have a set of cars $C$, where $\abs{C} \in [1, 10^3]$. Each car $c \in C$ has a given path $\bm{p_c}$ through the city. The path is a sequence of streets the car has to drive through.
The number of streets in each path $\bm{p_c}$ is in the range $[2, 10^3]$.
No intersection or street can be repeated in a path.

At the beginning of the simulation, all cars are at the end of the first street in their path. They either wait if the light is red or are ready to move if the light is green. If more cars start at the end of the same street, they queue up according to their IDs in the input file (see Figure~\ref{fig:hashcode_street}). When a car reaches the end of the last street in its path, it is immediately removed from the street.

\begin{figure}[ht] % h = here, t = top, p = page of floats
    \centering
    \includegraphics[width=\linewidth]{img/hashcode/figure3.png}
    %\includegraphics[width=.8\linewidth]{img/hashcode/figure3.png}
    \caption[Example of cars driving through a street]{
        This figure shows the first five seconds of a~simulation \cite{google2023google}.
        For simplicity, only one street is shown in the figure; when the light is red for this street, it is green for another street in the same intersection.
        When the light turns green at $T=1s$, the first (yellow) car immediately passes the intersection and moves to the next street, reaching the end of the street at $T=4s$.
        At $T=2s$, the light is still green, so the second (red) car passes the intersection and moves to the next street, reaching the end of the street at $T=5s$.
        From $T=3s$ to $T=5s$, the light is red, so the third (purple) car cannot pass the intersection and has to wait for the next cycle.
    }
    \label{fig:hashcode_street}
\end{figure}

\subsection{Score}

The score reflects the quality of the traffic light schedules: the more cars reach their destination, and the sooner they arrive, the higher the score.
The objective is to maximize this value.
In the context of optimization, the score serves as a \textit{fitness function}.

The score is determined as follows. Let $\bm{\theta}$ denote the traffic light schedules, and let $C$ be the set of cars.
Given a simulation duration $\mathrm{D} \in [1, 10^4]$ in seconds,
and a fixed bonus awarded for reaching the destination $\mathrm{F} \in [1, 10^3]$,
let $t(c; \bm{\theta}) \in \mathbb{N}$ be the time when a car $c \in C$ reaches its destination.
The score of a single car $c$ under the schedules $\bm{\theta}$ is defined as
\begin{equation}
    score(c; \bm{\theta}) =
    \begin{cases}
        \mathrm{F} + (\mathrm{D} - t(c; \bm{\theta})), & \text{if $t(c; \bm{\theta}) \leq \mathrm{D}$}, \\
        0, & \text{otherwise}.
    \end{cases}
\end{equation}
Then, the total score of the schedules $\bm{\theta}$ is defined as
\begin{equation}
    SCORE(\bm{\theta}) = \sum_{c \in C} score(c; \bm{\theta}).
\end{equation}

\section{Preprocessing of a problem instance} \label{sec:preprocessing}

In this section, we introduce preprocessing steps that we apply to reduce the total number of parameters we need to optimize.
These are our own observations and are not part of the original problem statement. After preprocessing, a substantial amount of intersections and streets is removed from optimization, which allows us to save resources and focus only on the important parameters. We also describe the format in which we represent the schedules.

\subsection{Preprocessing steps}
Throughout this section, we introduce some additional terms to help us better describe the steps.

\paragraph{Used and unused streets}
\textit{Unused street} is a street that is the final destination of all cars that have it in their path (i.e., a traffic light for this street is not needed).
\textit{Used street} is a street that is not the final destination of at least one car that has it in its path.

\paragraph{Used and unused intersections} \textit{Unused intersection} is an intersection where all incoming streets are unused streets. \textit{Used intersection} is an intersection with at least one used incoming street.

% https://tex.stackexchange.com/questions/32160/new-line-after-paragraph
\paragraph{Step 1: Remove unused intersections and unused streets} \label{para:step_1} \mbox{} \\
Remove unused intersections and unused incoming streets from used intersections. \\

\hyperref[para:step_1]{Step 1} is an obvious one; unused intersections are never used and only add unnecessary complexity. Unused incoming streets in used intersections prolong the traffic light cycle for the whole intersection, which very frequently leads to longer waiting times and thus a lower score. We now proceed to the next step.

\paragraph{Trivial and non-trivial intersections}
There are two types of used intersections: trivial and non-trivial.
\textit{Trivial intersection} is an intersection with exactly one used incoming street. \textit{Non-trivial intersection} is an intersection with two or more used incoming streets.

\paragraph{Trivial and non-trivial streets}
For completeness, we also split the used streets into trivial and non-trivial ones.
\textit{Trivial street} is a used incoming street in a trivial intersection. \textit{Non-trivial street} is a used incoming street in a non-trivial intersection.

\paragraph{Step 2: Fix the schedule for trivial intersections; optimize only schedules of non-trivial intersections} \label{para:step_2}
For each trivial intersection, the schedule can be fixed by assigning a green light to its only used incoming street for the entire duration. Trivial intersections can then be removed from optimization, and only schedules of non-trivial intersections are optimized. \\

Let us think about \hyperref[para:step_2]{Step 2} in more detail. For trivial intersections, we have two meaningful schedule options:
\begin{enumerate}
    \item Keep the light green the whole time.
    \item Keep the light red the whole time, effectively blocking all cars there.
\end{enumerate}
Empirically speaking, the vast majority of trivial intersections should be kept green, otherwise many cars would not be able to reach their destination at all.

However, suppose that keeping the light red for some trivial intersection improves the score. This means that there must be a problematic car passing through this intersection. Such a car must eventually reach some non-trivial intersection via a street, and this street can still be set to have a red light the whole time during optimization. We therefore leave this up to the optimization algorithm, hoping that it explores this option if it is indeed beneficial. \\

To demonstrate the usefulness of preprocessing, Tables~\ref{tab:intersection_statistics} and \ref{tab:street_statistics} show statistics of intersections and streets for each dataset. After applying both steps, we are only left with non-trivial intersections and non-trivial streets for optimization. This enables us to skip optimizing thousands of parameters.

% TABLE OF INTERSECTION STATISTICS
\begin{table}[h]
\centering\footnotesize\sf

\begin{tabular}{lr@{\hspace{0.1cm}}r@{\hspace{0.5cm}}r@{\hspace{0.1cm}}r@{\hspace{0.5cm}}r@{\hspace{0.1cm}}r}

& \multicolumn{6}{c}{\textbf{Intersections}} \\
\addlinespace[2pt]
\toprule
Dataset & \multicolumn{2}{c}{Unused} & \multicolumn{2}{c}{Trivial} & \multicolumn{2}{c}{Non-trivial} \\
\midrule
\textbf{A} & 1 & (25\%) & 2 & (50\%) & \textbf{1} & \textbf{(25\%)} \\
\textbf{B} & 777 & (11\%) & 4,977 & (70\%) & \textbf{1,319} & \textbf{(19\%)} \\
\textbf{C} & 2,340 & (23\%) & 4,468 & (45\%) & \textbf{3,192} & \textbf{(32\%)} \\
\textbf{D} & 0 & (0\%) & 0 & (0\%) & \textbf{8,000} & \textbf{(100\%)} \\
\textbf{E} & 0 & (0\%) & 263 & (53\%) & \textbf{237} & \textbf{(47\%)} \\
\textbf{F} & 30 & (2\%) & 332 & (20\%) & \textbf{1,300} & \textbf{(78\%)} \\
\bottomrule
\end{tabular}

\caption[Intersection statistics]{Intersection statistics across all datasets.}
\label{tab:intersection_statistics}
\end{table}

% TABLE OF STREET STATISTICS
\begin{table}[h]
\centering\footnotesize\sf

\begin{tabular}{lr@{\hspace{0.1cm}}r@{\hspace{0.5cm}}r@{\hspace{0.1cm}}r@{\hspace{0.5cm}}r@{\hspace{0.1cm}}r}

& \multicolumn{6}{c}{\textbf{Streets}} \\
\addlinespace[2pt]
\toprule
Dataset & \multicolumn{2}{c}{Unused} & \multicolumn{2}{c}{Trivial} & \multicolumn{2}{c}{Non-trivial} \\
\midrule
\textbf{A} & 1 & (20\%) & 2 & (40\%) & \textbf{2} & \textbf{(40\%)} \\
\textbf{B} & 1,138 & (12\%) & 4,977 & (55\%) & \textbf{2,987} & \textbf{(33\%)} \\
\textbf{C} & 23,558 & (67\%) & 4,468 & (13\%) & \textbf{7,004} & \textbf{(20\%)} \\
\textbf{D} & 12,054 & (13\%) & 0 & (0\%) & \textbf{83,874} & \textbf{(87\%)} \\
\textbf{E} & 42 & (4\%) & 263 & (26\%) & \textbf{693} & \textbf{(70\%)} \\
\textbf{F} & 4,667 & (47\%) & 332 & (3\%) & \textbf{5,001} & \textbf{(50\%)} \\
\bottomrule
\end{tabular}

\caption[Street statistics]{Street statistics across all datasets.}
\label{tab:street_statistics}
\end{table}

From now on, whenever we refer to schedules, we mean the \textit{schedules of non-trivial intersections with non-trivial streets}, as these intersections and streets are the only ones that are left after preprocessing.

\subsection{Schedules format} \label{subsec:schedules_format}

Each schedule of an intersection consists of two parts: \textit{order} and \textit{times}. Order is an array of street indices that defines the order in which the streets have the green light. Times is an array of integers that defines the duration of the green light with respect to the street order.
For illustration, recall the schedule from Figure~\ref{fig:hashcode_traffic_lights}; street-a (ID 0) has a green light for 2 seconds, and then street-b (ID~1) has a green light for 3 seconds. This schedule is represented in our format as follows:
\vspace{-0.4cm}
\begin{align*}
\text{Order} \quad& \text{Times} \\
\Big([0, 1], \quad& [2, 3]\Big).
\end{align*}
Then, the schedules are collectively represented as
\begin{quote}
    a \textbf{list of pairs}, where each pair consists of \textbf{order} and \textbf{times} arrays.
\end{quote}

\section{Datasets} \label{sec:datasets}

This section briefly describes all datasets provided with the problem.
The datasets vary in the size and structure of their city plans.
In the previous section, we already showed statistics of intersections and streets (see Tables~\ref{tab:intersection_statistics} and \ref{tab:street_statistics}).
Here we compare the datasets by the number of parameters we need to optimize.
That is, twice the number of non-trivial streets---one parameter for the order and one for the time. Note that this is only one way to compare the datasets, and it does not account for the actual paths cars must take or how convoluted those paths may be.

\begin{figure}[h]
    \centering
    % \includegraphics[width=\linewidth]{img/hashcode/figure5.png}
    % \includegraphics[width=.6\linewidth]{img/hashcode/figure5.png}
    \includegraphics[width=.5\linewidth]{img/hashcode/figure5.png}
    \caption[Visualization of dataset A]{
        Visualization of dataset A \cite{google2023google}.
    }
    \label{fig:hashcode_dataset_a}
\end{figure}

\paragraph{A - An example: 4 parameters} Simple toy problem dataset used for debugging (see Figure~\ref{fig:hashcode_dataset_a}).

\paragraph{B - By the ocean: 5,974 parameters} Dataset based on a real city plan of Lisbon, Portugal (see Figure~\ref{fig:hashcode_dataset_b}).

\begin{figure}[h]
    \centering
    \includegraphics[width=\linewidth]{img/screenshots/hashcode_datasets_b.png}
    % \includegraphics[width=.8\linewidth]{img/screenshots/hashcode_datasets_b.png}
    \caption[Visualization of dataset B]{
        Dataset B based on the real data of Lisbon on the right. Screenshot from \href{https://www.youtube.com/watch?v=YPOVd-hQUjA}{Hash Code 2021: Online Qualification Round Livestream}.
    }
    \label{fig:hashcode_dataset_b}
\end{figure}

\paragraph{C - Checkmate: 14,008 parameters} Dataset with a chessboard-like pattern and regular structure of intersections and streets (see Figure~\ref{fig:hashcode_dataset_c_e}).

% https://codeforces.com/blog/entry/88188#comment-765574
\paragraph{D - Daily commute: 167,748 parameters} By far the largest dataset with a challenging-to-navigate network from the \textit{Barabási-Albert} distribution \cite{albert2002statistical}.

\paragraph{E - Etoile: 1,386 parameters} Nicknamed \textit{Etoile}\footnote{\textit{Étoile} means star in French.}, this dataset is a one big star, meaning there is one very important intersection in the middle with hundreds of incoming streets (see Figure~\ref{fig:hashcode_dataset_c_e}).

\paragraph{F - Forever jammed: 10,002 parameters} Medium sized dataset but again with a complex network difficult to optimize.

\begin{figure}[h]
    \centering
    \includegraphics[width=\linewidth]{img/screenshots/hashcode_datasets_c_e.png}
    % \includegraphics[width=.8\linewidth]{img/screenshots/hashcode_datasets_c_e.png}
    \caption[Visualization of datasets C and E]{
        Visualization of datasets C and E. Screenshot from \href{https://www.youtube.com/watch?v=YPOVd-hQUjA}{Hash Code 2021: Online Qualification Round Livestream}.
    }
    \label{fig:hashcode_dataset_c_e}
    \end{figure}

\subsection{Score normalization} \label{subsec:score_normalization}

As mentioned in Section~\ref{sec:competition_overview}, each dataset yields an absolute score within a different range. To compare the performance across all datasets, we normalize the scores to a 0--1 scale, where 0 represents our baseline solution (see Section~\ref{sec:initial_schedules} for details), and 1 corresponds to the maximum known score\footnote{See the maximum known scores at \\\url{https://github.com/sagishporer/hashcode-2021-qualification\#score}.} for the dataset. Note that the baseline is already a good solution, so there may be limited room for improvement---for example, in dataset B.

\section{Simulator} \label{sec:simulator}

This section provides a brief overview of the custom simulator that we developed to serve as a black-box fitness function for the optimization algorithms. We discuss the motivation behind its design and describe the functionality relevant to the optimization process.
For further information on using the simulator or regarding its implementation, please refer to the user guide in Appendix~\ref{chap:user_guide} and the developer documentation in Appendix~\ref{chap:developer_documentation}.

\medskip

Our simulator replaces the original competition evaluation system, which was accessible only through the competition website via a user interface and is no longer available. It also adds several features on top of the original system.
The simulator is written in C++ to be as fast and efficient as possible, and is wrapped as a Python package using the powerful pybind11 library~\cite{jakob2017pybind11}, making it very convenient to use.
Inspired by libraries such as NumPy\footnote{\url{https://numpy.org/}} and PyTorch\footnote{\url{https://pytorch.org/}}, our goal is to provide an easy-to-use interface and seamless integration with the vital Python ecosystem, without compromising on top-tier performance.

Following are the features of the simulator package that are important for optimization. These features are implemented in C++ and exposed through a lightweight Python API:
\begin{itemize}
    \item Reading and storing the input data.
    \item Creating initial schedules using several different options listed in Section~\ref{sec:initial_schedules}.
    \item Running the simulation to evaluate given schedules.
    \item Retrieving and updating schedules in a format described in Section~\ref{subsec:schedules_format}.
\end{itemize}

The only case where larger data are passed between C++ and Python is when retrieving and updating the schedules.
Since the optimization loop is implemented in Python using parts of the DEAP library~\cite{fortin2012deap}, it must retrieve the schedules from the simulator, modify them externally, and then send them back into the simulator to evaluate their score.

\include{chap02}
\chapter{Theory of optimization methods} \label{chap:optimization_methods}

In this chapter, we cover the theory of the methods we use to optimize the Traffic signaling problem. Specifically, we describe \textit{hill climbing} (Section~\ref{sec:hill_climbing}), \textit{simulated annealing} (Section~\ref{sec:simulated_annealing}) and \textit{genetic algorithm} (Section~\ref{sec:genetic_algorithm}). In Section~\ref{sec:initialization} and Section~\ref{sec:parallelization} we discuss common methods for initializing and parallelizing the algorithms.

In optimization, the goal is to find the best state according to an objective function, also known as a \textit{fitness function}. In other words, the fitness function is a problem-specific function that defines how well a solution solves a particular problem.
Because the state space of optimized problems can be very large or even infinite, systematic search methods are often unsuitable. Instead, we can use the so-called \textit{heuristic algorithms}. These algorithms do not guarantee to find the~best solution but are frequently used in practice because they can find good solutions in reasonable time. Our three chosen methods are all examples of heuristic algorithms.

\section{Hill climbing} \label{sec:hill_climbing}

Hill climbing \cite{russell2020artificial, luke2013essentials} is one of the simplest local search algorithms. It keeps track of the current state and on each iteration moves to the neighboring state with the highest value, i.e., it moves in the direction that provides the biggest improvement. If there are no neighboring states with a higher value, the algorithm terminates. The greedy nature of hill climbing is both its strength and its weakness. It can make rapid progress towards a good solution but once it gets stuck in a local optimum it cannot escape it.

A lot of variants of this algorithm exist. In our experiments, we use a version called \textit{first-choice hill climbing} (see Algorithm~\ref{alg:first_choice_hill_climbing}). This version randomly generates next states until it finds one with a higher value than the current state and then moves to this state. This strategy works well when there are many neighboring states or when no defined neighborhood structure exists, which is exactly our case.

\begin{algorithm}
% https://tex.stackexchange.com/questions/150184/how-to-avoid-uppercase-function-name-while-using-function
%\algrenewcommand\textproc{} % disable uppercase function name

\begin{algorithmic}
\caption{First-choice hill climbing} \label{alg:first_choice_hill_climbing}
\Function{First-Choice-Hill-Climbing}{$problem$}
\State $current \gets problem$.I\textsc{nitial}
\For{$t = 1$ \textbf{to} M\textsc{ax\_iterations}}
    \State $next \gets$ a randomly generated next state
    \If{$\Call{Value}{next} > \Call{Value}{current}$}
        \State $current \gets next$
    \EndIf
\EndFor
\State \Return $current$
\EndFunction
\end{algorithmic}
\end{algorithm}

\section{Simulated annealing} \label{sec:simulated_annealing}

Simulated annealing \cite{russell2020artificial, luke2013essentials} builds up on the idea of hill climbing and solves the problem of getting stuck in local optima (see Algorithm~\ref{alg:simulated_annealing}). In each iteration, it randomly generates a next state. If the state is better, it is always accepted. Otherwise, the worse state is accepted with some probability. The probability decreases exponentially with how much worse the state is and with the current value of the ``cooling'' schedule. By allowing moves to worse states, the algorithm can escape local optima and therefore find better solutions.

The idea of the cooling schedule is inspired by the process of annealing in metallurgy, where metals are heated to a high temperature and then slowly cooled down to reach a strong, crystalline structure. Analogously, we want to allow worse moves more often at the beginning during the exploration phase and then gradually decrease that chance to hopefully converge to the best solution. Obviously, the choice of the schedule is absolutely crucial for the performance of the algorithm and has to be tuned specifically for each problem.

\begin{algorithm}
    % https://tex.stackexchange.com/questions/150184/how-to-avoid-uppercase-function-name-while-using-function
    %\algrenewcommand\textproc{} % disable uppercase function name
    
    \begin{algorithmic}
    \caption{Simulated annealing} \label{alg:simulated_annealing}
    \Function{Simulated-Annealing}{$problem$, $schedule$}
    \State $current \gets problem$.I\textsc{nitial}
    \For{$t = 1$ \textbf{to} M\textsc{ax\_iterations}}
        \State $T \gets$ \Call{$schedule$}{$t$}
        \State $next \gets$ a randomly generated next state
        \State $\Delta E \gets \Call{Value}{next} - \Call{Value}{current}$
        \If{$\Delta E > 0$}
            \State $current \gets next$
        \Else
            \State $current \gets next$ with probability $e^{\Delta E / T}$
        \EndIf
    \EndFor
    \State \Return $current$
    \EndFunction
    \end{algorithmic}
\end{algorithm}

\section{Genetic algorithm} \label{sec:genetic_algorithm}

% https://martinpilat.com/en/nature-inspired-algorithms/evolutionary-algorithms-introduction
% https://martinpilat.com/en/nature-inspired-algorithms/evolutionary-algorithms-continuous-and-combinatorial-optimization

Initially proposed by Holland \cite{holland1975adaptation} and Goldberg \cite{goldberg1989genetic}, genetic algorithm \cite{russell2020artificial,vanneschi2023lectures} belongs to the group of \textit{evolutionary algorithms}. Those algorithms are inpired by the seminal Darwin's theory of evolution, particularly by the idea of \textit{natural selection} \cite{darwin1859origin}. It states that the individuals with better traits suited to their environment are more likely to adapt, survive and reproduce. Over successive generations, these advantageous traits become more common in the population, leading to the emergence of new species.

This biological motivation is translated into the context of evolutionary algorithms in the following way. We work with a set of individuals called a \textit{population}. Each individual in this population represents a solution to a given problem. The algorithm runs in iterations called \textit{generations}. In each generation, it selects some individuals, modifies them using the so-called \textit{genetic operators} and then selects which individuals survive to the next generation. The selection process prefers individuals that are better at solving the problem, i.e., individuals with higher fitness.
There are many different evolutionary algorithms with even more variations that follow the previous description. Let us now focus specifically on the genetic algorithm (see Algorithm~\ref{alg:genetic_algorithm}).

\begin{algorithm}
    % https://tex.stackexchange.com/questions/150184/how-to-avoid-uppercase-function-name-while-using-function
    %\algrenewcommand\textproc{} % disable uppercase function name
    
    \begin{algorithmic}
    \caption{Genetic algorithm} \label{alg:genetic_algorithm}
    \Function{Genetic-Algorithm}{$population$}
    \State $population \gets \Call{Initialize}{population}$
    \For{$t = 1$ \textbf{to} M\textsc{ax\_iterations}}
    \State $population \gets \Call{Evaluate}{population}$
    \State $selected \gets \Call{Select}{population}$
    % Bad formatting of ff in math mode
    % https://tex.stackexchange.com/questions/6046/bad-formating-of-ff-in-math-mode
    \State $\mathit{offspring} \gets \Call{Reproduce}{selected}$
    \State $population \gets \mathit{offspring}$
    \EndFor
    \State \Return $population$
    \EndFunction
    \\
    \Function{Reproduce}{$selected$}
    \State $\mathit{offspring} \gets$ empty list
    \For{$parent1$, $parent2$ \textbf{in} $selected$}
    \State $child1$, $child2 \gets$ \Call{Crossover}{$parent1$, $parent2$}
    % algorithmic: one line if
    % https://tex.stackexchange.com/questions/624457/algorithmic-put-if-else-and-endif-into-the-same-line
    % \State\algorithmicif\ small probability \algorithmicthen\ $child1 \gets \Call{Mutate}{child1}$
    % \State\algorithmicif\ small random probability \algorithmicthen\ $child2 \gets \Call{Mutate}{child2}$
    \If{small random probability}
    \State $child1 \gets \Call{Mutation}{child1}$
    \EndIf
    \If{small random probability}
    \State $child2 \gets \Call{Mutation}{child2}$
    \EndIf
    \State add $child1$, $child2$ to $\mathit{offspring}$
    \EndFor
    \State \Return $\mathit{offspring}$
    \EndFunction
\end{algorithmic}
\end{algorithm}

\subsection{Selection}

For the selection of parents for the next generation, common methods include the \textit{roulette wheel selection} and the \textit{tournament selection} \cite{vanneschi2023lectures}. Roulette wheel selection selects individuals with a probability directly proportional to their fitness. Its downside is that it does not work for negative fitness and can be sensitive to changes in fitness values. Tournament selection works by repeatedly sampling a few individuals and selecting the best one among them. Thus, it is not dependent on specific values and can be used with any fitness function.

\subsection{Crossover and mutation}

The offspring is created from the selected parents using genetic operators \textit{crossover} and \textit{mutation}. Crossover combines two parents to produce two new children. There are many different ways to perform crossover, the most common being the \textit{one-point crossover}. Since individuals are usually represented as strings or arrays of numbers, one-point crossover simply picks a point in the individual and creates two new individuals by copying the first part from one parent and the second part from the other parent and vice versa. This can be extended to the \textit{two-point crossover} or the \textit{k-point crossover} in general. Other examples of crossover, e.g., if an individual represents a permutation and simple swapping of parts cannot be used, include the \textit{order crossover (OX)} or the \textit{partially mapped crossover (PMX)} \cite{vanneschi2023lectures}, also known as the \textit{partially matched crossover (PMX)}.

Mutation is a small random change in an individual. It is used to introduce variability into the population and to prevent the algorithm from getting stuck in local optima. Mutation is usually more problem-specific than crossover and is often customized to the problem at hand. Some common examples are the \textit{bit-flip mutation} for binary strings, the \textit{uniform} and the \textit{gaussian} mutations, i.e., a mutation replacing a value with a number drawn from one of these distributions, or the \textit{index shuffle} mutation for permutations.

\bigskip

The created offspring directly replaces the old population and the process repeats. Unfortunately, this mechanism does not guarantee that the best individual will be preserved in the next generation. To prevent this, a technique called \textit{elitism} can be used. Elitism always transfers a few of the best parents to the next generation, ensuring that the best solution found so far is not lost. The rest of the population is then filled with the offspring as previously mentioned.

\section{Initialization} \label{sec:initialization}

In all of the previously mentioned algorithms, we start with some initial state or population of initial states. The simplest and most straightforward way is to generate the initial state randomly. This is often good enough, but if the state space is large or the problem is complex, the algorithm can be very inefficient. Instead, we can try to come up with some smarter heuristic method to start in a more promising region of the search space. This is especially beneficial if we have some prior knowledge about the problem because we can help the algorithm skip exploring irrelevant parts of the search space, effectively reducing the number of iterations needed to find a good solution. An example of such a heuristic initialization is the \textit{nearest neighbors} initialization for the \textit{Traveling Salesman Problem (TSP)}, where the next node in the sequence is chosen as the closest unvisited node instead of generating the sequence randomly.

However, using heuristics for initialization can be a double-edged sword as it can lead to \textit{premature convergence} \cite{vanneschi2023lectures}. This means losing the diversity of solutions too early and getting stuck in a local optimum.
When using optimization algorithms, it is always important to balance the trade-off between \textit{exploration} and \textit{exploitation} \cite{luke2013essentials}. Exploration is a process of searching the search space for new solutions, while exploitation is a process of refining the current solutions.

\section{Parallelization} \label{sec:parallelization}

As hinted in the previous sections, heuristic algorithms can run for a long time. It is not uncommon to compute hundreds of thousands of fitness evaluations---imagine a genetic algorithm with a population size of 200, run for 1000 generations. These evaluations can take a while, especially if they are simulations, games, neural network training, or similar. As a result, we may want to use parallelization to speed up this process and fully utilize our computational resources.

Single-state methods like hill climbing and simulated annealing are inherently sequential and thus difficult to parallelize. But we can at least run multiple independent instances of the algorithm in parallel. Note that this is different from methods like \textit{beam search}, where useful information is shared between the instances.

For population methods like genetic algorithm, parallelization makes much more sense. The most common way is to run the fitness evaluations of individuals in parallel. This is great because the evaluations are completely independent of each other and the fitness computation is often the biggest bottleneck of the whole algorithm. Other parts, such as mutation and crossover, can also be parallelized across individuals, but these are usually much faster than the evaluation and may not provide much speedup, if any.

\chapter{Application of optimization algorithms} \label{chap:algorithms_application}

In this chapter, we describe the problem-specific details of our chosen optimization algorithms. Section~\ref{sec:initial_schedules} presents different options for creating initial schedules and explains which ones were chosen for optimization.
Section~\ref{sec:modifying_the_schedules} gives a high-level overview of how the schedules are modified by the algorithms.
Sections~\ref{sec:genetic_algorithm_application},~\ref{sec:hill_climbing_application}, and~\ref{sec:simulated_annealing_application} cover the exact configuration of genetic algorithm, hill climbing, and simulated annealing, respectively, including their hyperparameters. Section~\ref{sec:hyperparameter_search} explains how we searched for the best hyperparameters and summarizes the final settings for each algorithm and dataset.

\section{Initial schedules} \label{sec:initial_schedules}

As mentioned in Section~\ref{sec:simulator}, the simulator offers several options for generating initial schedules.
Here we describe the different options for initializing both \textit{order} and \textit{times} (see Section~\ref{subsec:schedules_format}). The order initialization options include:
\begin{itemize}
    \item \textit{default} - simply uses the order given by street IDs in the input file
    \item \textit{random} - takes a random permutation of the streets
    \item \textit{adaptive} - determines the order during a simulation run---each street is assigned to the earliest free position in the order array when used for the first time (only compatible with the \textit{default times} initialization; with other options, it may result in an inconsistent format and fail to generate schedules)
\end{itemize}
The times initialization options include:
\begin{itemize}
    \item \textit{default} - all times are set to 1 second
    \item \textit{scaled} - time for each street is a total number of cars using this street divided by a single given constant (the divisor)
\end{itemize}

\medskip

Both \textit{order initialization} and \textit{times initialization} are hyperparameters.
To determine the best settings for optimization, we experimentally compared the performance of different initialization options for each dataset.
Specifically, we compared the following options:
\begin{itemize}
    \item \textcolor{myblue}{\textbf{default (baseline)}} - default order and default times
    \item \textcolor{myorange}{\textbf{adaptive}} - adaptive order and default times
    \item \textcolor{mygreen}{\textbf{random}} - random order and default times
    \item \textcolor{myred}{\textbf{scaled}} - default order and scaled times
\end{itemize}
Scaled option uses the best divisor between 1--100 for each dataset.
Random option is stochastic and is therefore averaged over 100 trials.

\begin{figure}[h]
    \centering
    % \includegraphics[width=.8\linewidth]{img/screenshots/hashcode_datasets_c_e.png}
    \includegraphics[width=\linewidth]{img/experiments/pdfa-init_experiment.pdf}
    \caption[Comparison of initialization options]{
        Comparison of different initialization options for each dataset.
        Y-axis shows the normalized score (see Section~\ref{subsec:score_normalization}).
        The black error bars indicate the 95\% confidence intervals.
    }
    \label{fig:init_comparison}
\end{figure}

The results are shown in Figure~\ref{fig:init_comparison}.
The \textcolor{mygreen}{\textbf{random}} option performs similarly to the \textcolor{myblue}{\textbf{default}} option---since the default order is neither optimized nor intentionally poor, it is reasonable that the random order has comparable performance on average.
The \textcolor{myorange}{\textbf{adaptive}} option achieves strong results, especially on datasets B, C, and D. This is likely because streets in these datasets are mostly used by one or a few cars, making the specific order very important and the default times a suitable choice.
The \textcolor{myred}{\textbf{scaled}} option performs best on datasets E and F. Unsure about the reason for dataset E, but dataset F contains many streets used by hundreds of cars, so it makes sense to increase the green light time for these streets.

We see that using other ``smarter'' initialization methods can significantly improve the baseline solution, enabling the optimization process to begin from a much better starting point.
We decided to use \textit{adaptive order} and \textit{default times} for datasets B, C, and D, and \textit{random order} and \textit{scaled times} for datasets E and F (see Table~\ref{tab:hyperparams_datasets_specific}).

\section{Modifying the schedules} \label{sec:modifying_the_schedules}

Once the initial schedules are created, we can start modifying them using the proposed optimization algorithms.
Since the schedules format (see Section~\ref{subsec:schedules_format}) is quite complex and each intersection's schedule may have a different number of streets, operations such as crossover or mutation are applied separately for each intersection. Moreover, because the \textit{times} array depends on the \textit{order} array, any modifications to the order array must also be applied to the corresponding times array.

\section{Genetic algorithm} \label{sec:genetic_algorithm_application}

In this section, we apply the theory introduced in Section~\ref{sec:genetic_algorithm} and describe our design of the genetic algorithm, including the selection, crossover, and mutation operators, as well as the corresponding hyperparameters. We begin with this method, as the other algorithms reuse its mutation operator to generate new states.

\subsection{Selection}

For selecting individuals for reproduction, we use \textit{tournament selection} combined with \textit{elitism}. The corresponding hyperparameters are \textit{tournament size} and \textit{elitism}, which defines the percentage of top-performing individuals that are directly selected for reproduction.
\subsection{Crossover}

For each intersection, we randomly decide whether to crossover only the order, only the times, or both. We use \textit{order crossover (OX)} for modifying the order, and \textit{two-point crossover} for modifying the times. Crossover is performed with a probability defined by the hyperparameter \textit{crossover probability}.

\subsection{Mutation} \label{sec:mutation_application}

For each intersection, we randomly decide whether to mutate only the order, only the times, or both. We use \textit{index shuffle} for modifying the order, and for the times, we add or subtract one to some values at random. Mutation is performed with a probability defined by the hyperparameter \textit{mutation probability}. When mutation is applied, the hyperparameter \textit{mutation bit rate} controls how likely each individual value is to be modified. It can be given either as a probability or as an integer specifying the expected number of modified values in the state.

\bigskip

The remaining hyperparameters of the genetic algorithm are \textit{population size} and \textit{generations}---their names are quite self-explanatory. Together with the crossover and mutation probabilities, these hyperparameters control the total number of fitness evaluations (i.e., simulation runs) performed by the algorithm.
Additionally, utilizing the approach from Section~\ref{sec:parallelization}, we run fitness evaluations within each population in parallel.

\section{Hill climbing} \label{sec:hill_climbing_application}

As explained in Section~\ref{sec:hill_climbing}, we use a variant of the algorithm that generates next states randomly because there is no explicit neighborhood structure. We simply apply the mutation operator from the genetic algorithm (see Section~\ref{sec:mutation_application}) to generate the next state. Note that we again use the hyperparameter \textit{mutation bit rate} but not the mutation probability because the mutation is always applied. The only other hyperparameter is \textit{iterations}, which sets the number of iterations the algorithm runs for.

\section{Simulated annealing} \label{sec:simulated_annealing_application}

Introduced in Section~\ref{sec:simulated_annealing}, this algorithm uses the same strategy to generate next states as presented in Section~\ref{sec:hill_climbing_application}. That means it has the same hyperparameters \textit{mutation bit rate} and \textit{iterations}. The only additional hyperparameter is \textit{initial temperature}, which sets the initial temperature of the cooling schedule. For the cooling schedule, we use a linear decay. It is defined as
\begin{equation}
    schedule(t) = \tau_{init} \cdot \left(1 - \frac{t}{T}\right) + \varepsilon,
\end{equation}
where $\tau_{init}$ is the \textit{initial temperature}, $t$ is the current iteration, and $T$ is the total number of \textit{iterations}.
% Both of these values are hyperparameters, with the initial temperature especially requiring careful tuning for each dataset to achieve good results.

\section{Hyperparameter search} \label{sec:hyperparameter_search}

Performance of all three aforementioned algorithms largely depends on the setting of their hyperparameters.
To find the best hyperparameters for the experiments, we use a form of \textit{greedy search}. That is, we only focus on optimizing one hyperparameter at a time and try to find the best value for it. Other hyperparameters are currently fixed---either heuristically or set to their already optimized values.
We test a number of reasonable values for each hyperparameter and perform runs with additional values if the results are not satisfactory.
Each setting is tested on 10 different fixed seeds and the results are averaged.

To maintain comparability, we allocate an equal budget of fitness evaluations to all three algorithms. For hill climbing and simulated annealing, this budget is easily enforced via the \textit{iterations} hyperparameter. For genetic algorithm, the number of evaluations is stochastic and cannot be precisely set; however, we always set the \textit{generations} hyperparameter so that the expected total number of fitness evaluations matches the predefined budget.

For each dataset, we use the best initialization options that we selected in Section~\ref{sec:initial_schedules}.

% TABLE WITH TESTED HYPERPARAMETER VALUES
\begin{table}[h]
\centering\footnotesize\sf

\begin{tabular}{l@{\hspace{0.5cm}}c}
% \toprule
% \multicolumn{2}{l}{\textbf{Tested values}} \\
Crossover probability & $\{0.2, 0.3, \ldots, 0.8\}$ \\
Mutation probability & $\{0.1, 0.2, \ldots, 1.0\}$ \\
Tournament size & $\{2, 3, \ldots, 10\}$ \\
Elitism & $\{0, 0.05, 0.1, \ldots, 0.3\}$ \\
Population size & $\{10, 20, \ldots, 100\}$ \\
% \midrule
Mutation bit rate & $\{1, 2, \ldots, 20\}$ \\
Initial temperature & many values between $0.1$ and $500$ \\
% \bottomrule
\end{tabular}

\caption[Tested hyperparameter values]{Ranges of tested hyperparameter values.}
\label{tab:hyperparams_tested_values}
\end{table}

To give a clearer picture of the values explored, Table~\ref{tab:hyperparams_tested_values} summarizes the ranges used in the hyperparameter search.
The genetic algorithm parameters---\textit{crossover probability}, \textit{mutation probability}, \textit{tournament size}, and \textit{elitism}---were tested only on smaller datasets E and B to reduce search complexity, because they seemed generalizable across datasets. Other parameters, especially the \textit{mutation bit rate} and \textit{initial temperature}, are highly dataset-dependent and were tuned separately for each algorithm and dataset.

Finally, Tables~\ref{tab:hyperparams_shared} and~\ref{tab:hyperparams_datasets_specific} provide a detailed overview of the selected hyperparameters for the experiments. For clarity, the first table lists the hyperparameters that are shared across all datasets, while the second table shows dataset-specific hyperparameters.

% TABLE WITH SHARED HYPERPARAMETERS
\begin{table}[h]
\centering\footnotesize\sf

\begin{tabular}{l@{}c}
% \toprule
\multicolumn{2}{l}{\textbf{Genetic Algorithm}} \\
Crossover probability & 0.6 \\
Mutation probability & 0.4 \\
Tournament size & 3 \\
Elitism & 0.05 \\
\midrule
\multicolumn{2}{l}{\textbf{Hill Climbing}} \\
Iterations & 450,000 \\
\midrule
\multicolumn{2}{l}{\textbf{Simulated Annealing}} \\
Iterations & 450,000 \\
% \bottomrule
\end{tabular}

\caption[Shared hyperparameters]{Shared hyperparameters for all datasets.}
\label{tab:hyperparams_shared}
\end{table}

% TABLE WITH DATASET-SPECIFIC HYPERPARAMETERS
\begin{table}[hb!]
\centering\footnotesize\sf

\begin{minipage}[t]{0.48\textwidth}
\centering
\begin{tabular}{l@{\hspace{0.5cm}}c}
% \toprule
\multicolumn{2}{c}{\textbf{Dataset E}} \\
\midrule
Order initialization & random \\
Times initialization & scaled \\
\midrule
\multicolumn{2}{l}{\textbf{Genetic Algorithm}} \\
Population size & 90 \\
Generations & 6,667 \\
Mutation bit rate & 9 \\
\midrule
\multicolumn{2}{l}{\textbf{Hill Climbing}} \\
Mutation bit rate & 15 \\
\midrule
\multicolumn{2}{l}{\textbf{Simulated Annealing}} \\
Mutation bit rate & 5 \\
Initial temperature & 275 \\
% \bottomrule
\end{tabular}
\end{minipage}
\hfill
\begin{minipage}[t]{0.48\textwidth}
\centering
\begin{tabular}{l@{\hspace{0.5cm}}c}
% \toprule
\multicolumn{2}{c}{\textbf{Dataset B}} \\
\midrule
Order initialization & adaptive \\
Times initialization & default \\
\midrule
\multicolumn{2}{l}{\textbf{Genetic Algorithm}} \\
Population size & 10 \\
Generations & 60,000 \\
Mutation bit rate & 2 \\
\midrule
\multicolumn{2}{l}{\textbf{Hill Climbing}} \\
Mutation bit rate & 4 \\
\midrule
\multicolumn{2}{l}{\textbf{Simulated Annealing}} \\
Mutation bit rate & 1 \\
Initial temperature & 0.25 \\
% \bottomrule
\end{tabular}
\end{minipage}
\newline
\newline
\newline
\begin{minipage}[t]{0.48\textwidth}
\centering
\begin{tabular}{l@{\hspace{0.5cm}}c}
% \toprule
\multicolumn{2}{c}{\textbf{Dataset F}} \\
\midrule
Order initialization & random \\
Times initialization & scaled \\
\midrule
\multicolumn{2}{l}{\textbf{Genetic Algorithm}} \\
Population size & 50 \\
Generations & 12,000 \\
Mutation bit rate & 2 \\
\midrule
\multicolumn{2}{l}{\textbf{Hill Climbing}} \\
Mutation bit rate & 5 \\
\midrule
\multicolumn{2}{l}{\textbf{Simulated Annealing}} \\
Mutation bit rate & 3 \\
Initial temperature & 15 \\
% \bottomrule
\end{tabular}
\end{minipage}
\hfill
\begin{minipage}[t]{0.48\textwidth}
\centering
\begin{tabular}{l@{\hspace{0.5cm}}c}
% \toprule
\multicolumn{2}{c}{\textbf{Dataset C}} \\
\midrule
Order initialization & adaptive \\
Times initialization & default \\
\midrule
\multicolumn{2}{l}{\textbf{Genetic Algorithm}} \\
Population size & 20 \\
Generations & 30,000 \\
Mutation bit rate & 2 \\
\midrule
\multicolumn{2}{l}{\textbf{Hill Climbing}} \\
Mutation bit rate & 4 \\
\midrule
\multicolumn{2}{l}{\textbf{Simulated Annealing}} \\
Mutation bit rate & 2 \\
Initial temperature & 0.1 \\
% \bottomrule
\end{tabular}
\end{minipage}
\newline
\newline
\newline
\begin{minipage}[t]{0.48\textwidth}
\centering
\begin{tabular}{l@{\hspace{0.5cm}}c}
% \toprule
\multicolumn{2}{c}{\textbf{Dataset D}} \\
\midrule
Order initialization & adaptive \\
Times initialization & default \\
\midrule
\multicolumn{2}{l}{\textbf{Genetic Algorithm}} \\
Population size & 40 \\
Generations & 15,000 \\
Mutation bit rate & 2 \\
\midrule
\multicolumn{2}{l}{\textbf{Hill Climbing}} \\
Mutation bit rate & 2 \\
\midrule
\multicolumn{2}{l}{\textbf{Simulated Annealing}} \\
Mutation bit rate & 2 \\
Initial temperature & 0.1 \\
% \bottomrule
\end{tabular}
\end{minipage}

\caption[Dataset-specific hyperparameters]{
    Dataset-specific hyperparameters.
    The order and times initializations are the same for all three methods for each dataset.
}
\label{tab:hyperparams_datasets_specific}
\end{table}

\chapter{Experimental results} \label{chap:experimental_results}

This chapter presents the results of our experiments. Specifically, we compare the performance of three optimization algorithms:
\begin{itemize}
    \item \textit{genetic algorithm (GA)},
    \item \textit{hill climbing (HC)},
    \item \textit{simulated annealing (SA)},
\end{itemize}
on datasets B, C, D, E, and F.
Section~\ref{sec:experimental_setup} explains the setup of the experiments and format of the results shown in the figures and tables. The following sections present the results for each dataset, including plots and tables summarizing the performance of each algorithm. The datasets are presented in ascending order of their size (see Section~\ref{sec:datasets} for more details).

\section{Experimental setup} \label{sec:experimental_setup}

As previously mentioned, Tables~\ref{tab:hyperparams_shared} and \ref{tab:hyperparams_datasets_specific} summarize the hyperparameters used for each algorithm and dataset. The experiments were performed on either AMD EPYC 9454 or AMD EPYC 9474F CPUs. Each setting was run with 10 different fixed seeds, and the results were averaged to ensure reliability. As explained in Section~\ref{sec:hyperparameter_search}, all three algorithms were run for the same fixed budget of fitness evaluations.

Note that comparing by number of evaluations is somewhat disadvantageous for the genetic algorithm, which inherently performs a broader search of the state space. Additionally, fitness evaluations within each population are computed in parallel, which can result in significantly faster runtimes compared to single-state methods. Nevertheless, we chose the number of fitness evaluations as the comparison metric instead of runtime, as it better reflects algorithmic efficiency and is independent of implementation details and the specific hardware used.

For each dataset, we provide a plot illustrating the performance of all three methods throughout the optimization process. The x-axis indicates the number of fitness evaluations, while the primary y-axis shows the normalized score (see Section~\ref{subsec:score_normalization}).
For completeness, the secondary y-axis displays the absolute score.
For genetic algorithm, the plotted curve represents the best score within the current population. For hill climbing and simulated annealing, the curve shows the current score at each step.  All curves are averaged over 10 runs, with the shaded areas indicating the standard deviation.

For each dataset, we also include a table reporting the overall best score achieved by each algorithm, averaged over 10 runs, along with the total runtime. The runtime is included to give a general idea of how long each method took to run, but it should not be interpreted as an exact metric.

Finally, we summarize in Table~\ref{tab:best_scores} the best scores achieved by each algorithm in a single run (out of 10 seeds), and compare them to the maximum known score for each dataset. The schedules corresponding to the best scores achieved for each dataset during optimization are included in the thesis attachment.

\newpage
\section{Dataset E} \label{sec:dataset_e}

For dataset E, the smallest optimized dataset, \textit{SA} not only outperformed \textit{GA} and \textit{HC}, but even exceeded the best known score, which explains why its normalized score in Table~\ref{tab:dataset_e_results} is greater than 1. While \textit{GA} and \textit{HC} quickly reached good scores above 80\% of the best known score, they converged early and improved only slightly afterwards. In contrast, \textit{SA} started off slower and volatile, but continued to steadily improve its score throughout the optimization process (see Figure~\ref{fig:dataset_e_experiment}).

\bigskip

\begin{figure}[h]
    \centering
    % \includegraphics[width=\linewidth]{img/screenshots/hashcode_datasets_c_e.png}
    % \includegraphics[width=.8\linewidth]{img/screenshots/hashcode_datasets_c_e.png}
    \includegraphics[width=\linewidth]{img/experiments/pdfa-e_Genetic_Algorithm_Hill_Climbing_Simulated_Annealing.pdf}
    \caption[Performance of the algorithms on dataset E]{
        Performance of the algorithms on dataset E.
    }
    \label{fig:dataset_e_experiment}
\end{figure}

\bigskip

\begin{table}[h]
\centering\footnotesize\sf
\begin{tabular}{lccc}
\toprule
& Normalized Score & Score & Runtime (mm:ss) \\
\midrule
\textcolor{myblue}{\textbf{Genetic Algorithm}} & 0.88 & 767,768 & \textbf{06:58} \\
\textcolor{myorange}{\textbf{Hill Climbing}} & 0.87 & 766,798 & 07:03 \\
\textcolor{mygreen}{\textbf{Simulated Annealing}} & \textbf{1.01} & \textbf{780,299} & 07:10 \\
\bottomrule
\end{tabular}
\caption[Statistics for dataset E]{
    Final statistics for dataset E.
}
\label{tab:dataset_e_results}
\end{table}

\newpage
\section{Dataset B} \label{sec:dataset_b}

For dataset B, \textit{SA} again achieved the best score among the three algorithms. This time, it performed the best since the very beginning of the optimization process (see Figure~\ref{fig:dataset_b_experiment}). Although close, it fell short of the best known score, reaching 97\% of it (see Table~\ref{tab:dataset_b_results}). \textit{HC} started off similarly to \textit{SA}, but again converged early and ultimately performed the worst---though it still achieved a good score. \textit{GA} performed noticeably better than \textit{HC} but did not reach the performance of \textit{SA}. Note that the parallel fitness evaluation in \textit{GA} starts to show its advantage, with the \textit{GA} runtime being approximately one-third shorter than the other two algorithms (see Table~\ref{tab:dataset_b_results}).

\bigskip

\begin{figure}[h]
    \centering
    % \includegraphics[width=\linewidth]{img/screenshots/hashcode_datasets_c_e.png}
    % \includegraphics[width=.8\linewidth]{img/screenshots/hashcode_datasets_c_e.png}
    \includegraphics[width=\linewidth]{img/experiments/pdfa-b_Genetic_Algorithm_Hill_Climbing_Simulated_Annealing.pdf}
    \caption[Performance of the algorithms on dataset B]{
        Performance of the algorithms on dataset B.
    }
    \label{fig:dataset_b_experiment}
\end{figure}

\bigskip

\begin{table}[h]
\centering\footnotesize\sf
\begin{tabular}{lccc}
\toprule
& Normalized Score & Score & Runtime (h:mm:ss) \\
\midrule
\textcolor{myblue}{\textbf{Genetic Algorithm}} & 0.91 & 4,570,095 & \textbf{0:42:29} \\
\textcolor{myorange}{\textbf{Hill Climbing}} & 0.87 & 4,569,945 & 1:00:08 \\
\textcolor{mygreen}{\textbf{Simulated Annealing}} & \textbf{0.97} & \textbf{4,570,309} & 0:59:07 \\
\bottomrule
\end{tabular}
\caption[Statistics for dataset B]{
    Final statistics for dataset B.
}
\label{tab:dataset_b_results}
\end{table}

\newpage
\section{Dataset F} \label{sec:dataset_f}

For dataset F, the performance of all three algorithms is much closer than in previous datasets, due to starting from a really good initial solution. Nonetheless, \textit{SA} achieved the highest score, \textit{HC} performed the worst, and \textit{GA} was in between (see Table~\ref{tab:dataset_f_results}). As shown in Figure~\ref{fig:dataset_f_experiment}, the shaded areas representing standard deviation overlap quite a lot, indicating that there is not a big difference between the algorithms. \textit{GA} again demonstrated a significantly shorter runtime.

\bigskip

\begin{figure}[h]
    \centering
    % \includegraphics[width=\linewidth]{img/screenshots/hashcode_datasets_c_e.png}
    % \includegraphics[width=.8\linewidth]{img/screenshots/hashcode_datasets_c_e.png}
    \includegraphics[width=\linewidth]{img/experiments/pdfa-f_Genetic_Algorithm_Hill_Climbing_Simulated_Annealing.pdf}
    \caption[Performance of the algorithms on dataset F]{
        Performance of the algorithms on dataset F.
    }
    \label{fig:dataset_f_experiment}
\end{figure}

\bigskip

\begin{table}[h]
\centering\footnotesize\sf
\begin{tabular}{lccc}
\toprule
& Normalized Score & Score & Runtime (h:mm:ss) \\
\midrule
\textcolor{myblue}{\textbf{Genetic Algorithm}} & 0.93 & 1,437,086 & \textbf{1:48:33} \\
\textcolor{myorange}{\textbf{Hill Climbing}} & 0.93 & 1,434,129 & 2:41:04 \\
\textcolor{mygreen}{\textbf{Simulated Annealing}} & \textbf{0.94} & \textbf{1,440,097} & 2:28:03 \\
\bottomrule
\end{tabular}
\caption[Statistics for dataset F]{
    Final statistics for dataset F.
}
\label{tab:dataset_f_results}
\end{table}

\newpage
\section{Dataset C} \label{sec:dataset_c}

For dataset C, the curves shown in Figure~\ref{fig:dataset_c_experiment} resemble those observed for dataset B (see Figure~\ref{fig:dataset_b_experiment}). \textit{SA} consistently outperformed the other algorithms from the start and again exceeded the best known score, with the normalized score above 1 (see Table~\ref{tab:dataset_c_results}). \textit{GA} and \textit{HC} performed similarly, with \textit{GA} narrowly outperforming \textit{HC}. As in previous datasets, \textit{GA} finished in a significantly shorter time (see Table~\ref{tab:dataset_c_results}).

\bigskip

\begin{figure}[h]
    \centering
    % \includegraphics[width=\linewidth]{img/screenshots/hashcode_datasets_c_e.png}
    % \includegraphics[width=.8\linewidth]{img/screenshots/hashcode_datasets_c_e.png}
    \includegraphics[width=\linewidth]{img/experiments/pdfa-c_Genetic_Algorithm_Hill_Climbing_Simulated_Annealing.pdf}
    \caption[Performance of the algorithms on dataset C]{
        Performance of the algorithms on dataset C.
    }
    \label{fig:dataset_c_experiment}
\end{figure}

\bigskip

\begin{table}[h]
\centering\footnotesize\sf
\begin{tabular}{lccc}
\toprule
& Normalized Score & Score & Runtime (h:mm:ss) \\
\midrule
\textcolor{myblue}{\textbf{Genetic Algorithm}} & 0.94 & 1,314,347 & \textbf{2:02:52} \\
\textcolor{myorange}{\textbf{Hill Climbing}} & 0.93 & 1,314,200 & 3:33:10 \\
\textcolor{mygreen}{\textbf{Simulated Annealing}} & \textbf{1.01} & \textbf{1,315,476} & 3:26:31 \\
\bottomrule
\end{tabular}
\caption[Statistics for dataset C]{
    Final statistics for dataset C.
}
\label{tab:dataset_c_results}
\end{table}

\newpage
\section{Dataset D} \label{sec:dataset_d}

For dataset D, the largest dataset by far, the results are different from all previous datasets. \textit{HC}, the weakest algorithm in the previous cases, performed the best here, although only slightly ahead of \textit{SA} (see Figure~\ref{fig:dataset_d_experiment}). It is likely that the algorithms would have benefited from running for more evaluations, as their performance curves had not yet fully converged.
\textit{GA}, in particular, performed the worst here, but completed in less than half the time of the other two algorithms (see Table~\ref{tab:dataset_d_results}), fully utilizing its parallel evaluation.
However, the runtimes for this dataset were already much longer than for the rest of the datasets combined, showing how large and demanding dataset D really was.

\bigskip

\begin{figure}[h]
    \centering
    % \includegraphics[width=\linewidth]{img/screenshots/hashcode_datasets_c_e.png}
    % \includegraphics[width=.8\linewidth]{img/screenshots/hashcode_datasets_c_e.png}
    \includegraphics[width=\linewidth]{img/experiments/pdfa-d_Genetic_Algorithm_Hill_Climbing_Simulated_Annealing.pdf}
    \caption[Performance of the algorithms on dataset D]{
        Performance of the algorithms on dataset D.
    }
    \label{fig:dataset_d_experiment}
\end{figure}

\bigskip

\begin{table}[h]
\centering\footnotesize\sf
\begin{tabular}{lccc}
\toprule
& Normalized Score & Score & Runtime (hh:mm:ss) \\
\midrule
\textcolor{myblue}{\textbf{Genetic Algorithm}} & 0.90 & 2,508,730 & \textbf{09:41:52} \\
\textcolor{myorange}{\textbf{Hill Climbing}} & \textbf{0.92} & \textbf{2,525,531} & 20:59:59 \\
\textcolor{mygreen}{\textbf{Simulated Annealing}} & 0.92 & 2,522,204 & 22:39:17 \\
\bottomrule
\end{tabular}
\caption[Statistics for dataset D]{
    Final statistics for dataset D.
}
\label{tab:dataset_d_results}
\end{table}

\begin{table}
\centering\footnotesize\sf

\begin{tabular}{lr@{\hspace{0.5cm}}r@{\hspace{0.5cm}}r@{\hspace{0.5cm}}r}
\toprule
Dataset & \textcolor{myblue}{\textbf{GA}} & \textcolor{myorange}{\textbf{HC}} & \textcolor{mygreen}{\textbf{SA}} & \textit{Max known score} \\
\midrule
\textbf{B} & 4,570,168 & 4,569,994 & 4,570,346 & \textit{4,570,431} \\
\textbf{C} & 1,314,597 & 1,314,584 & \textbf{1,315,702} & \textit{1,315,372} \\
\textbf{D} & 2,512,355 & 2,528,954 & 2,525,797 & \textit{2,610,027} \\
\textbf{E} & 771,025 & 768,443 & \textbf{782,044} & \textit{779,279} \\
\textbf{F} & 1,440,172 & 1,439,639 & 1,443,333 & \textit{1,480,489} \\
\bottomrule
\end{tabular}

\caption[Best scores]{
    Best scores achieved in a single run (out of 10 seeds) by each algorithm, compared to the max known score. Bold values indicate newly achieved best scores that outperform the max known score. The schedules corresponding to the best scores achieved for each dataset during optimization are included in the thesis attachment.
}
\label{tab:best_scores}
\end{table}


% \begin{table}[b!]

% \centering
% %% Tabulka používá následující balíčky:
% %%   - booktabs (\toprule, \midrule, \bottomrule)
% %%   - dcolumn (typ sloupce D: vycentrovaná čísla zarovnaná na
% %%     desetinnou čárku
% %%     Všimněte si, že ve zdrojovém kódu jsou desetinné tečky, ale
% %%     tisknou se čárky.
% %% Dále používáme příkazy \pulrad a \mc definované v makra.tex

% \begin{tabular}{l@{\hspace{1.5cm}}D{.}{,}{3.2}D{.}{,}{1.2}D{.}{,}{2.3}}
% \toprule
%  & \mc{} & \mc{\textbf{Směrod.}} & \mc{} \\
% \pulrad{\textbf{Efekt}} & \mc{\pulrad{\textbf{Odhad}}} & \mc{\textbf{chyba}$^a$} &
% \mc{\pulrad{\textbf{P-hodnota}}} \\
% \midrule
% Abs. člen     & -10.01 & 1.01 & \mc{---} \\
% Pohlaví (muž) & 9.89   & 5.98 & 0.098 \\
% Výška (cm)    & 0.78   & 0.12 & <0.001 \\
% \bottomrule
% \multicolumn{4}{l}{\footnotesize \textit{Pozn:}
% $^a$ Směrodatná chyba odhadu metodou Monte Carlo.}
% \end{tabular}

% \caption{Maximálně věrohodné odhady v~modelu M.}\label{tab03:Nejaka}

% \end{table}

% \begin{table}
% % uncomment the following line if you use the fitted top captions for tables
% % (see the \floatsetup[table] comments in `macros.tex`.
% %\floatbox{table}[\FBwidth]{
% \centering\footnotesize\sf
% \begin{tabular}{llrl}
% \toprule
% Column A & Column 2 & Numbers & More \\
% \midrule
% Asd & QWERTY & 123123 & -- \\
% Asd qsd 1sd & \textcolor{red}{BAD} & 234234234 & This line should be helpful. \\
% Asd & \textcolor{blue}{INTERESTING} & 123123123 & -- \\
% Asd qsd 1sd & \textcolor{violet!50}{PLAIN WEIRD} & 234234234 & -- \\
% Asd & QWERTY & 123123 & -- \\
% \addlinespace % a nice non-intrusive separator of data groups (or final table sums)
% Asd qsd 1sd & \textcolor{green!80!black}{GOOD} & 234234299 & -- \\
% Asd & NUMBER & \textbf{123123} & -- \\
% Asd qsd 1sd & \textcolor{orange}{DANGEROUS} & 234234234 & (no data) \\
% \bottomrule
% \end{tabular}
% %}{  % uncomment if you use the \floatbox (as above), erase otherwise
% \caption{An example table.  Table caption should clearly explain how to interpret the data in the table. Use some visual guide, such as boldface or color coding, to highlight the most important results (e.g., comparison winners).}
% %}  % uncomment if you use the \floatbox
% \label{tab:z}
% \end{table}


\chapter*{Conclusion}
\addcontentsline{toc}{chapter}{Conclusion}

In this thesis, we addressed the Traffic signaling problem from the Google Hash Code competition, which serves as a simplified version of the real-world problem of traffic signal optimization. We began by implementing a fast and efficient simulator in C++, which we wrapped as a Python package. We then integrated the simulator into a Python optimization pipeline as a black-box fitness function. This setup enabled quick evaluation of solutions and allowed us to perform many iterations of our three chosen optimization algorithms: \textit{Genetic Algorithm (GA)}, \textit{Hill Climbing (HC)}, and \textit{Simulated Annealing (SA)}. We then compared the performance of these algorithms on the provided competition datasets of different sizes and structures.

Our experiments showed that all three algorithms achieved good results across all datasets. However, \textit{SA} consistently outperformed the others on every dataset except the largest one, dataset D, where the otherwise weakest \textit{HC} performed slightly better---likely because the algorithms had not yet fully converged. Furthermore, \textit{SA} was able to surpass the max known scores in datasets C and E, achieving new best results. \textit{GA} generally performed better than \textit{HC}, but only by a small margin.
The performance of all algorithms was highly dependent on the choice of hyperparameter values.

The superior performance of \textit{SA} compared to \textit{HC} is unsurprising, as \textit{SA} is a more capable algorithm in theory. Its ability to move to worse states obviously broadens the search and helps to escape local optima. Nonetheless, based on our experimental experience, we believe that simply always moving to a state with the same fitness value as the current state is the key factor behind \textit{SA}'s success.

On the other hand, \textit{GA}'s performance was somewhat below expectations, especially considering the additional complexity involved compared to the single-state methods. There are several possible reasons for this. As previously hinted, comparing the algorithms by the number of evaluations is disadvantageous for \textit{GA}. Moreover, greedily optimizing one hyperparameter at a time may be less optimal for \textit{GA} because it has more hyperparameters---some form of grid search could be more appropriate. Additionally, the initial population might have lacked diversity or been completely homogeneous. Unfortunately, we were unable to come up with initializations that would increase diversity and still achieve good results.
Lastly, we think that our optimization problem does not benefit much from the broader search capabilities of \textit{GA} and is rather suited to methods that refine a single solution.

For future work, we could try to improve the performance of \textit{GA} by exploring additional initializations, performing a wider hyperparameter grid search, or using some form of informed crossover or mutation. It would also be interesting to run \textit{GA} for the same amount of time as \textit{SA} to see if it can catch up. This approach could maybe better reflect the real-world scenario where we are constrained by time and aim to fully utilize \textit{GA}'s parallel fitness evaluation.
Additionally, we could test other single-state optimization methods such as \textit{Iterated Local Search}~\cite{lourenco2018iterated} or \textit{Tabu Search}~\cite{glover1998tabu}.


%%% Bibliography
\include{bibliography}

%%% Figures used in the thesis (consider if this is needed)
% \listoffigures

%%% Tables used in the thesis (consider if this is needed)
%%% In mathematical theses, it could be better to move the list of tables to the beginning of the thesis.
% \listoftables

%%% Abbreviations used in the thesis, if any, including their explanation
%%% In mathematical theses, it could be better to move the list of abbreviations to the beginning of the thesis.
% \chapwithtoc{List of abbreviations}

% \begin{description}
% 	\item[GA] genetic algorithm
% 	\item[HC] hill climbing
% 	\item[SA] simulated annealing 
% \end{description}

% \printglossary[title={List of Abbreviations}]

%%% Doctoral theses must contain a list of author's publications
\ifx\ThesisType\TypePhD
\chapwithtoc{List of Publications}
\fi

%%% Attachments to the thesis, if any. Each attachment must be referred to
%%% at least once from the text of the thesis. Attachments are numbered.
%%%
%%% The printed version should preferably contain attachments, which can be
%%% read (additional tables and charts, supplementary text, examples of
%%% program output, etc.). The electronic version is more suited for attachments
%%% which will likely be used in an electronic form rather than read (program
%%% source code, data files, interactive charts, etc.). Electronic attachments
%%% should be uploaded to SIS. Allowed file formats are specified in provision
%%% of the rector no. 72/2017. Exceptions can be approved by faculty's coordinator.
\appendix
% \chapter{Attachments}

% \section{First Attachment}

\chapter{User guide} \label{chap:user_guide}

This guide is intended for users interested in running the code associated with this thesis. It is logically divided into two parts:
\begin{itemize}
    \item \textbf{Simulator} - a standalone tool for evaluating the Traffic signaling problem (described in Chapter~\ref{chap:problem_description}) with various handy features, wrapped into the \verb|traffic-signaling| Python package
    \item \textbf{Optimization and experiments} - scripts for running the optimization and experiments, which use the simulator
\end{itemize}

\section{Prerequisites}

When listing version requirements for the tools below, we specify the \textit{minimum} supported version. As of July 2025, the latest available versions (e.g., Python 3.13) are also compatible. However, we cannot guarantee compatibility with all future versions.

\bigskip

To build and install the \verb|traffic-signaling| package, you need:
\begin{itemize}
    \item \textbf{C++ compiler} with C++20 code support; e.g., gcc 11, clang 16, or MSVC 19.29 (or later versions)
    \item \textbf{Python} 3.10 or later
    \item \textbf{Python headers} (most likely already installed with Python)
    \begin{itemize}
        \item You can verify that the headers are available by inspecting their expected location with e.g.
\begin{verbatim}
    python3 -c "import sysconfig;
        print(sysconfig.get_path('include'))"
\end{verbatim}
        \item If not available, install the headers with e.g.
\begin{verbatim}
    sudo apt install python3-dev
\end{verbatim}
    \end{itemize}
\end{itemize}
To easily set up the environment for optimization and run the experiments, you further need:
\begin{itemize}
    \item \textbf{GNU Make}
\end{itemize}
Optionally, if you want to run all unit tests for both C++ and Python, you additionally need:
\begin{itemize}
    \item \textbf{Git}
    \item \textbf{CMake} 3.24 or later
\end{itemize}

\newpage

\section{Simulator}

\textbf{If you only want to run the optimization and do not want to use the simulator as a standalone tool, feel free to skip this section.}

\bigskip

After satisfying the prerequisites, you can simply build and install \\
the \verb|traffic-signaling| Python package by running the following command in the top-level directory:
\begin{verbatim}
    pip install ./traffic_signaling
\end{verbatim}
This will install the package into your Python environment (preferably into a virtual environment), making it available for use in your Python scripts.

\bigskip

To run Python unit tests for the package using Python's \verb|unittest| built-in framework, use the following command:
\begin{verbatim}
    make test_package_python
\end{verbatim}
To run both C++ and Python unit tests using CMake, use the following command:
\begin{verbatim}
    make test_package_cmake
\end{verbatim}
Note that running the make commands will create a virtual environment \verb|.venv| and install all required packages there using pip.

\subsection{Code example}

The following code snippet demonstrates some of the package's functionality. 
For the full API reference, see the \verb|traffic_signaling/traffic_signaling| subdirectory, which contains the files \verb|city_plan.pyi|, \verb|simulation.pyi|, and \verb|utils.py|. These files include extensive docstrings for every class and method, effectively serving as the package documentation.

\begin{lstlisting}[language=Python]
from traffic_signaling import *

# Load the city plan for a specific dataset
plan = create_city_plan(data='e')
# Create a simulation for the given city plan
sim = Simulation(plan)

# Initialize the traffic light schedules
sim.create_schedules(order='default', times='default')
# Save the schedules to a file in the competition format
sim.save_schedules('schedules.txt')
# Load the schedules from a file
sim.load_schedules('schedules.txt')

# Calculate the score
score = sim.score()
# Show a summary report of the simulation
sim.summary()
\end{lstlisting}

\bigskip

\section{Optimization} \label{sec:optimization}

\textbf{To quickly set up the environment for optimization, run the following command in the top-level directory:}
\begin{verbatim}
    make setup
\end{verbatim}
This will create a virtual environment \verb|.venv|, build and install the simulator and other necessary packages using pip into it, and compile \verb|operators.py| file using Cython\footnote{\url{https://cython.org/}} for better performance during optimization. Do not forget to activate the virtual environment before running any scripts with e.g.
\begin{verbatim}
    source .venv/bin/activate
\end{verbatim}
If you encounter any issues, try running \verb|make clean| and then \verb|make setup| again.

Now you can run the optimization algorithms using the \verb|optimizer.py| script. The script has two required positional arguments:
\begin{itemize}
    \item \verb|algorithm| - algorithm to use for optimization; possible values are \verb|ga|, \verb|hc|, \verb|sa|
    \item \verb|data| - input dataset to use; possible values are \verb|a|, \verb|b|, \verb|c|, \verb|d|, \verb|e|, \verb|f|
\end{itemize}
After specifying the required arguments, you can easily run the script with e.g.:
\begin{verbatim}
    python3 optimizer.py hc e
\end{verbatim}
This will run the Hill Climbing algorithm on dataset E using default values. However, you probably want to explicitly set some parameters, especially the hyperparameter values. If you prefer, you can run
\begin{verbatim}
    python3 optimizer.py --help
\end{verbatim}
to see the full usage. Below is a concise list of the parameters:
\begin{itemize}
    \item \verb|--order_init| - \textit{order initialization} hyperparameter; possible values are \verb|adaptive|, \verb|random|, \verb|default|
    \item \verb|--times_init| - \textit{times initialization} hyperparameter; possible values are \verb|scaled|, \verb|default|
    \item \verb|--mutation_bit_rate| - \textit{mutation bit rate} hyperparameter
    \item \verb|--population| - \textit{population size} hyperparameter (GA only)
    \item \verb|--generations| - \textit{generations} hyperparameter (GA only)
    \item \verb|--crossover| - \textit{crossover probability} hyperparameter (GA only)
    \item \verb|--mutation| - \textit{mutation probability} hyperparameter (GA only)
    \item \verb|--elitism| - \textit{elitism} hyperparameter (GA only)
    \item \verb|--tournsize| - \textit{tournament size} hyperparameter (GA only)
    \item \verb|--iterations| - \textit{iterations} hyperparameter (HC and SA)
    \item \verb|--temperature| - \textit{initial temperature} hyperparameter (SA only)
    \item \verb|--seed| - value of the random seed for reproducibility
    \item \verb|--threads| - number of threads for parallel evaluation
    \item \verb|--logdir| - custom name of the directory with results and logs
    \item \verb|--verbose| - whether to print detailed output during optimization
    \item \verb|--no-save| - skip saving results to the log directory
\end{itemize}
An example of running the script with more parameters could look like this:
\begin{verbatim}
    python3 optimizer.py ga e \
        --order_init random --times_init scaled \
        --mutation_bit_rate 5 --population 100 --generations 200 \
        --threads 16 --seed 21 --verbose
\end{verbatim}
When the optimization finishes (and if the \verb|--no-save| option was not used), the optimizer will save the following files in the log directory:
\begin{itemize}
    \item a CSV file containing statistics for each iteration / generation of the algorithm
    \item a file with the best schedules found, stored in the competition format
    \item a PDF file visualizing the optimization process
    \item an information file listing all parameters, their values, and additional details
\end{itemize}

\bigskip

Optionally, you can run unit tests for the optimizer with the following command:
\begin{verbatim}
    make test_optimizer
\end{verbatim}

\subsection{Running the experiments}

To replicate the experiments presented in this thesis, ensure that you have already run the \verb|make setup| command, then navigate to the \verb|experiments| directory. We recommend reading the \verb|experiments/README.md| file for more details, but for convenience, we also include the list of commands below. We strongly suggest running each algorithm and dataset separately.
\begin{itemize}
    \item \verb|make test| - run a simple sanity check experiment
    \item \verb|make init_experiment| - run a quick experiment comparing different initialization methods
    \item \verb|make run_{b,c,d,e,f}_{ga,hc,sa}| - run a specific algorithm on a specific dataset (using 10 runs with different fixed seeds)
    \item \verb|make run_{b,c,d,e,f}| - run all algorithms on a specific dataset
    \item \verb|make plot_{b,c,d,e,f}| - plot the results of a specific dataset
    \item \verb|make plots| - plot all datasets
    \item \verb|make all| - run everything and plot all results
\end{itemize}

The schedules corresponding to the best scores
achieved for each dataset during optimization are stored in the \verb|experiments/best_solutions| directory.


\chapter{Developer documentation} \label{chap:developer_documentation}

\section{Simulator}

The source code for the simulator is located in the \verb|traffic_signaling| directory. Below is an outline of the directory structure:
\begin{itemize}
  \item \texttt{include/} - C++ header files of the simulator
  \begin{itemize}
    \item \texttt{city\_plan/} - headers of \textit{city plan} part
    \item \texttt{simulation/} - headers of \textit{simulation} part
  \end{itemize}

  \item \texttt{src/} - C++ source files of the simulator
  \begin{itemize}
    \item \texttt{city\_plan/} - source files of \textit{city plan} part
    \item \texttt{simulation/} - source files of \textit{simulation} part
    \item \texttt{bindings/} - python bindings for \textit{city plan} and \textit{simulation} parts
  \end{itemize}

  \item \texttt{tests/} - Unit tests for both C++ and Python verifying the simulator functionality

  \item \texttt{traffic\_signaling/} - Contents of the Python package when installed with pip
  \begin{itemize}
    \item \texttt{utils.py} - provides extra utilities and helper functions for the simulator
    \item \texttt{data/} - contains the datasets provided with the competition
  \end{itemize}

  \item \texttt{pyproject.toml}, \texttt{setup.py} - Python configuration files for building and installing the \texttt{traffic-signaling} package using pip

  \item \texttt{CMakeLists.txt} - CMake configuration file for building the C++ code and the Python package using CMake
\end{itemize}
Both the C++ and Python sources are well documented with comments and docstrings to help you understand the code and its functionality.

\bigskip

The simulator is logically divided into two main parts:
\begin{itemize}
    \item \textit{city\_plan} - responsible for loading and storing input data; it essentially represents all ``static'' data known ahead of the simulation; its main class is \verb|CityPlan|
    \item \textit{simulation} - responsible for running the simulation and working with the traffic light schedules; it represents the ``dynamic'' data that change during the simulation; its main class is \verb|Simulation|
\end{itemize}

The motivation behind this separation is to initialize a single \verb|CityPlan| object at the beginning and reuse it across multiple simulations, possibly running in parallel---which is especially useful for the genetic algorithm.

\bigskip

These two parts are compiled into two Python extension modules---\verb|city_plan| and \verb|simulation|---using pybind11\footnote{\url{https://github.com/pybind/pybind11}}. They are then bundled together with the files in the \verb|traffic_signaling| subdirectory into the \verb|traffic-signaling| Python package. For more details about the bindings, see the \verb|bindings/simulation.cpp| and \verb|bindings/city_plan.cpp| files.

\bigskip

Note that when calling C++ code that takes a non-trivial amount of time to run from Python, we release the Python Global Interpreter Lock (GIL)\footnote{\url{https://docs.python.org/3/glossary.html\#term-global-interpreter-lock}}. This allows us to evaluate multiple simulations in parallel using regular threads without blocking the Python interpreter. It eliminates the need for slower and cumbersome multiprocessing, which is the typical approach to run parallel code in Python.

\subsection{Simulation algorithm}

Not only is the simulator implemented in C++ for performance reasons, but it also uses a \textbf{custom event queue algorithm} to ensure the simulation is evaluated as efficiently as possible. Rather than checking all cars every second of the simulation, we use a priority queue of street events sorted by their time of occurrence.

Let us briefly explain how the algorithm works.
As explained in Section~\ref{subsec:cars}, at the beginning of the simulation, all cars are at the end of the first street in their path, waiting for the green light. If there are more cars at the same street, they queue up according to their IDs.

We are at time $0$. For each car, we calculate the \textit{earliest time} it can get the green light on its current street and add it to the street's car queue.
The calculated time depends on the \textit{traffic light schedules} and \textit{other cars already waiting} in the car queue.
For the street, we add an event occurring at the calculated time to the event queue. The event indicates that the front car in the queue can now move to the next street.

Then, we iterate over the event queue until it is empty. We pop the first event from the queue and process it---that is, move the car to the next street and schedule another event at the time the car can get the green light on the next street. If the next street is the car's destination, we can immediately calculate the arrival time and remove the car from the simulation.

Using the event queue allows us to skip all unnecessary checks and reduce the operations to the required minimum, because we only process cars at the times when they are actually moving.
The main loop of the algorithm is implemented in the \verb|simulation/simulation.cpp| file in the \verb|Simulation::run| method. 

\section{Optimization}

The source code for optimization is in two files:
\begin{itemize}
    \item \verb|optimizer.py| - the main script providing the command-line interface for running the optimization algorithms 
    \item \verb|operators.py| - contains the implementation of the optimization algorithms and their operators
\end{itemize}
To implement the optimization algorithms, we used the DEAP\footnote{\url{https://github.com/deap/deap}} library, which provides a modular framework for evolutionary algorithms. However, we tweaked and rewrote some of its functions to suit our needs.

\bigskip

The \verb|optimizer.py| file contains the \verb|Optimizer| class, which is responsible for running the optimization (\verb|run| method). It processes the command-line arguments (described in~Section~\ref{sec:optimization}), initializes the traffic light schedules that we optimize, and runs the optimization algorithm. After the optimization is completed, the results and logs are saved to the specified log directory.

The fitness evaluation for schedules is implemented in the \verb|_evaluate| method.
The schedules initialization is implemented in the \verb|_create_individual| method.
All statistics and logs are handled together by the \verb|_save_statistics| method.

\bigskip

The \verb|operators.py| file contains the tweaked DEAP functions together with other custom functions such as \verb|crossover|, \verb|mutation|, \\
and \verb|tournament_selection_with_elitism|. \\
All three algorithms---\verb|genetic_algorithm|, \verb|hill_climbing|, \\
and \verb|simulated_annealing| functions---are modified versions of the original \verb|eaSimple| function from DEAP, which implements the simple genetic algorithm.

The file contains Cython type hints to enable compilation for faster performance during the optimization.


\end{document}
